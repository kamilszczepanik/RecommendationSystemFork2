\documentclass[../main.tex]{subfiles}

\begin{document}

Poniżej zamieszczono pełny opis wymagań projektu, składający się z wymagań funkcjonalnych i niefunkcjonalnych.

\subsection{Wymagania funkcjonalne}
\begin{itemize}
	\item System umożliwia rejestrację nowych użytkowników.
	\item System generuje rekomendacje filmów na podstawie recenzji i ocen użytkowników.
	\item Dane o filmach, studiach filmowych, aktorach, producentach są przechowywane i aktualizowane w systemie.
	\item System umożliwia zaawansowane wyszukiwanie filmów według różnych kryteriów.
	\item Użytkownicy mogą logować się do systemu za pomocą swoich danych uwierzytelniających.
	\item Użytkownicy mogą przeglądać filmy bez konieczności logowania.
	\item Użytkownicy mogą wejść w tryb inspekcji, aby zobaczyć dodatkowe szczegóły na temat filmu.
	\item Zalogowani użytkownicy mogą dodawać recenzje i oceny filmów.
	\item Zalogowani użytkownicy mogą dodawać filmy do ulubionych.
	\item Zalogowani użytkownicy mogą dodawać komentarze do recenzji.
	\item Zalogowani użytkownicy mogą edytować komentarze.
	\item Administrator systemu rozszerza klasę użytkownika.
	\item Administrator systemu ma możliwość usuwania użytkowników.
	\item Administrator systemu ma możliwość dodawania, edycji, usuwania filmów.
	\item Administrator systemu ma możliwość akceptacji oraz odrzucania zgłoszeń.
	\item Administrator systemu ma możliwość usuwania komentarzy.
\end{itemize}

\subsection{Wymagania niefunkcjonalne}
\begin{itemize}
	\item System musi obsługiwać wielu użytkowników jednocześnie.
	\item Czas dodawania recenzji nie powinien przekraczać 5 sekund.
	\item Czas generowania rekomendacji dla użytkownika nie powinien przekraczać 3 sekund.
	\item Czas ładowania strony z listą filmów nie powinien przekraczać 2 sekund.
	\item Maksymalna liczba wyświetlanych filmów na stronie to 50.
	\item Backup danych systemu jest tworzony codziennie.
	\item System musi być dostępny zarówno z systemów Linux, jak i Windows.
\end{itemize}

\end{document}
